\documentclass{article}
\usepackage{CJK} 
\usepackage{graphics}
\usepackage{pgf}
\usepackage{tikz}
\usetikzlibrary{calc,shadows}
\usetikzlibrary{decorations.markings,scopes}
\usetikzlibrary{arrows,snakes,backgrounds,shapes}
\usetikzlibrary{decorations.pathmorphing}
\newcommand{\blue}{\textcolor{blue}}
\newcommand{\red}{\textcolor{red}}
\newcommand{\purple}{\textcolor{purple}}


\pgfrealjobname{survey}
\begin{document}
\begin{CJK}{UTF8}{gkai} 
  \beginpgfgraphicnamed{DataStruct}
  \begin{tikzpicture}[scale=2,cap=round]

    % The graphic
    \node at (0,0)[fill=blue!20,draw,starburst,drop shadow,text width=4.5cm]
    {
      \blue{\large 数据结构}: 相互之间存在一种或多种特定关系的数据元素的集合.
    } ;

    \node at (0,-1.5) [fill=blue!20,draw,rectangle callout,, callout relative pointer={(0,0.7)},text width=6cm]
    {       
      计算机中,数据元素并不是孤立、杂乱无序的,而是具有内在联系的; \\ \vspace{0.1in}
      
      数据元素之间存在的一种或多种特定关系,就是数据的组织形式.
    };


  \end{tikzpicture}
  \endpgfgraphicnamed  
\end{CJK}

\end{document}
