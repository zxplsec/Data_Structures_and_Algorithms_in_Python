\documentclass{article}
\usepackage{CJK} 
\usepackage{graphics}
\usepackage{pgf}
\usepackage{tikz}
\usetikzlibrary{calc,shadows}
\usetikzlibrary{decorations.markings,scopes}
\usetikzlibrary{arrows,snakes,backgrounds,shapes}
\usetikzlibrary{decorations.pathmorphing}
\usepackage{listings}
\renewcommand{\ttdefault}{pcr}
\lstset{
  keywordstyle=\color{blue!70},
  frame=single,
  basicstyle=\ttfamily\bfseries\small,
  commentstyle=\small\color{red},
  rulesepcolor=\color{red!20!green!20!blue!20},
  tabsize=4,
  numbersep=5pt,
  %% backgroundcolor=\color{black!10},
  showspaces=false,
  showtabs=false,
  extendedchars=false,
  escapeinside=``,
  frame=no
}

\newcommand{\blue}{\textcolor{blue}}
\newcommand{\red}{\textcolor{red}}
\newcommand{\purple}{\textcolor{purple}}


\pgfrealjobname{survey}
\begin{document}
\begin{CJK}{UTF8}{gkai} 
  \beginpgfgraphicnamed{AlgEffAfter}
  \begin{tikzpicture}[node distance=4.5cm]
    \node [fill=blue!20,draw,starburst,drop shadow,text width=7.5cm] (1) {
      \blue{\large 事后统计方法}\\
      通过设计好的测试程序和数据,利用计算机计时器对不同算法编制的程序的运行时间进行比较,从而确定效率的高低。
    };

    \node[below of=1,rectangle split,rectangle split parts=4,rounded corners=3mm,draw,text width=10cm,fill=red!20] {
      \blue{缺点}
      \nodepart{second}
      须依据算法事先编制好程序
      \nodepart{third}
      时间的比较依赖于软硬件等环境因素\\
      \begin{itemize}
      \item 不同性能的机器上算法的表现不尽相同;
      \item 不同操作系统、编译器等也会影响算法的运行结果;\\
      \item CPU使用率和内存占用情况也会造成微小差异。
      \end{itemize}
      \nodepart{fourth}
      测试数据设计困难,且程序运行时间还与测试数据的规模有很大关系,
      效率高的算法在小的测试数据面前往往得不到体现。
    };

  \end{tikzpicture}
  \endpgfgraphicnamed  
\end{CJK}

\end{document}


