\documentclass{article}
\usepackage{CJK} 
\usepackage{graphics}
\usepackage{pgf}
\usepackage{tikz}
\usetikzlibrary{calc,shadows}
\usetikzlibrary{decorations.markings,scopes}
\usetikzlibrary{arrows,snakes,backgrounds,shapes}
\usetikzlibrary{decorations.pathmorphing}
\newcommand{\blue}{\textcolor{blue}}
\newcommand{\red}{\textcolor{red}}
\newcommand{\purple}{\textcolor{purple}}


\pgfrealjobname{survey}
\begin{document}
\begin{CJK}{UTF8}{gkai} 
  \beginpgfgraphicnamed{LogicPhySummary}
  \begin{tikzpicture}
  \node[copy shadow,fill=blue!20,draw=blue,thick,text width=7cm] at (3.5,0) {
    逻辑结构是面向问题的,而物理结构是面向计算机的,我们的目标是将数据及其逻辑关系存储到计算机的内存中。
  };
\end{tikzpicture}
  \endpgfgraphicnamed  
\end{CJK}

\end{document}


