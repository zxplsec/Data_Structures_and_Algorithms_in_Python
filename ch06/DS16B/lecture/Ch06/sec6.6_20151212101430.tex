\section{最短路径}
\begin{frame}\ft{\secname}
\begin{dingyi}
所谓最短路径,是指两点间经过的边上权值之和最小的路径。路径上的第一个顶点称为源点,最后一个顶点称为终点。
\end{dingyi}
\end{frame}

\begin{frame}\ft{\secname}
\begin{figure}
\centering
\input{Chapters/Ch06/Tikz/shortestpath}
\end{figure}•

\end{frame}


\subsection{\tf Dijkstra算法}
\begin{frame}\ft{\subsecname}
\begin{figure}
\centering
\input{Chapters/Ch06/Tikz/shortestpath1}
\end{figure}

$v_0$到$v_1$的最短距离为$1$,路径为$v_0\to v_1$。
\end{frame}



\begin{frame}\ft{\subsecname}
\lstinputlisting[
language=c,
linerange={121-131},
numbers=left,
numberstyle=\tiny,
]{Chapters/Ch06/Code/adjmatrix/adjmatrix.c}
\end{frame}

\begin{frame}\ft{\subsecname}

\end{frame}

\begin{frame}\ft{\subsecname}

\end{frame}

\begin{frame}\ft{\subsecname}

\end{frame}


\subsection{\tf Floyd算法}

\begin{frame}\ft{\subsecname}

\end{frame}

\begin{frame}\ft{\subsecname}

\end{frame}

\begin{frame}\ft{\subsecname}

\end{frame}

\begin{frame}\ft{\subsecname}

\end{frame}