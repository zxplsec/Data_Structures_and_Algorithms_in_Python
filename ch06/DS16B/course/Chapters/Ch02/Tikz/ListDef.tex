\documentclass{article}
\usepackage{CJK} 
\usepackage{graphics}
\usepackage{pgf}
\usepackage{tikz}
\usetikzlibrary{calc,shadows}
\usetikzlibrary{decorations.markings,scopes}
\usetikzlibrary{arrows,snakes,backgrounds,shapes}
\usetikzlibrary{decorations.pathmorphing}
\usepackage{listings}
\renewcommand{\ttdefault}{pcr}
\lstset{
  keywordstyle=\color{blue!70},
  frame=single,
  basicstyle=\ttfamily\bfseries\small,
  commentstyle=\small\color{red},
  rulesepcolor=\color{red!20!green!20!blue!20},
  tabsize=4,
  numbersep=5pt,
  %% backgroundcolor=\color{black!10},
  showspaces=false,
  showtabs=false,
  extendedchars=false,
  escapeinside=``,
  frame=no
}

\newcommand{\blue}{\textcolor{blue}}
\newcommand{\red}{\textcolor{red}}
\newcommand{\purple}{\textcolor{purple}}


\pgfrealjobname{survey}
\begin{document}
\begin{CJK}{UTF8}{gkai} 
  \beginpgfgraphicnamed{ListDef}
  \begin{tikzpicture}
    %The graphic
    \node at (0,0)[fill=blue!20,draw,starburst,drop shadow,text width=6cm]{
      \blue{\large 线性表(List)}\\
      零个或多个数据元素的有限序列。
    };

    \node at (0,-1.5) [below,fill=blue!20,draw,rectangle callout,callout relative pointer={(0.7,0.7)},rounded corners=3mm,text width=9cm]{
      若记线性表为$(a_1,\cdots,a_{i+1},a_i,a_{i+1},\cdots,a_n)$,则
      \begin{itemize}
      \item $n$为线性表的长度,当$n=0$时,称为空表;
      \item $a_1$称为首结点,$a_n$称为尾结点;
      \item $a_{i-1}$是$a_i$的直接前驱, $a_{i+1}$是$a_i$的直接后继。
      \item[] 当$i=1,2,\cdots,n-1$时,$a_i$有且仅有一个直接后继;
      \item[] 当$i=2,3,\cdots,n$时,$a_i$有且仅有一个直接前驱。
        
      \end{itemize}
    };
    
    %% \node at (0,-2.5) [fill=blue!20,draw,ellipse callout, callout relative pointer={(0.7,0.7)},text width=5cm]{ 
    %% 
    %% };
    
    %% \node at (0,-2.5) [text width=2cm,decorate,decoration=saw,fill=blue!20,draw,circle]{
    %% 
      
    %% };

  \end{tikzpicture}
  \endpgfgraphicnamed  
\end{CJK}

\end{document}


