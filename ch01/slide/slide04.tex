\section{什么是编程}

\begin{frame}\ft{\secname}
  编程是将算法转换为程序语言的过程,以便能被计算机所执行。

  \begin{itemize}
  \item 首先要有解决方案,亦即算法;
  \item 然后在选择合适的编程语言实现算法。
  \end{itemize}
  计算机科学不研究编程,但编程却是计算机科学家的重要能力。编程通常是解决方案的表达方式。
\end{frame}

\begin{frame}\ft{\secname}
  \begin{itemize}
  \item 算法描述了依据实际问题所生成的解决方案和产生预期结果所需要的一套步骤。

  \item 编程语言必须提供一种表示方法来表示对应的过程和数据。为此,它提供了\red{控制结构}和\red{数据类型}。
  \end{itemize}
\end{frame}

\begin{frame}\ft{\secname}
\red{控制结构}允许以方便而明确的方式表示算法步骤。至少,算法需要执行顺序处理、决策选择和控制迭代。只要语言提供这些基本语句,它就可以表达算法。
\end{frame}

\begin{frame}\ft{\secname}
  在计算机中,所有数据项都由一串一串的二进制数表示。为了让这些二进制串有意义,就需要有\red{数据类型}。 
  \begin{itemize}
  \item 数据类型为二进制数据提供解释,以便我们能够根据实际问题来思考数据。
    这些底层的内置数据类型(有时称为原始数据类型)为算法开发提供了基础。 \pause 
  \item[] 
    例如,大多数编程语言提供整数类型。内存中的二进制数据可解释为整数,并且给予一个与整数(如 23, 654 和 -19)相关联的含义。 \pause \\[0.1in]
  \item   此外,数据类型还提供数据项所参与操作的描述。对于整数,提供诸如加法、减法和乘法的操作。%我们期望数值类型的数据可以参与这些算术运算。
  \end{itemize}
% \end{frame}

% \begin{frame}\ft{\secname}
  
\end{frame}

\begin{frame}\ft{\secname}
  然而,通常我们遇到的困难是问题及其解决方案非常复杂。由语言提供的简单的结构和数据类型,虽然可以表示复杂的解决方案,但在实际中却不好用。
  我们需要一些方法控制这种复杂性,以助于形成更好的解决方案。
\end{frame}