\section{一个乱序字符串检查的例子}
\begin{frame}\ft{\secname}

  举一个“字符串乱序检查”的例子,以展示不同量级的算法。

  \begin{dingyi*}[乱序字符串]
    乱序字符串是指一个字符串只是另一个字符串的重新排列。
  \end{dingyi*}
  \vskip.1in \pause 

  \begin{li*}
    'heart' 和 'earth' 就是乱序字符串。'python' 和 'typhon'也是。
  \end{li*}
  \vskip.1in \pause

  为简单起见,假定所讨论的两个字符串长度相等,且都由 26 个小写字母组成。

  \blue{编写一个布尔函数,将两个字符串做参数并返回它们是不是乱序。}

\end{frame}
\subsection{方案一:检查}
\begin{frame}\ft{\subsecname}
  第一种方法是检查第一个字符串是不是出现在第二个字符串中。如果可以检验到每一个字符,那这两个字符串一定是乱序。 \pause 

  可以通过用 None 替换字符来了解一个字符是否完成检查。\pause

  但是,由于 Python 字符串是不可变的,所以第一步是将第二个字符串转换为列表。\pause

  检查第一个字符串中的每个字符是否存在于第二个列表中,如果存在,替换成 None。 
\end{frame}

\begin{frame}\ft{\subsecname}

  \lstinputlisting{code/anagramSolution1.py}  

\end{frame}

\begin{frame}\ft{\subsecname}

注意到 $s_1$ 的每个字符都会在 $s_2$ 中进行 $n$ 个字符的比较,比较次数为 
$n^2$。所以这个算法复杂度为 $O(n^2 )$。
\end{frame}

\subsection{方案二:排序和比较}
\begin{frame}\ft{\subsecname}
注意到:\red{即使 $s_1,s_2$ 不同,它们都是由完全相同的字符组成的。} 所以,可以按照字母顺序排列每个字符串,如果排列后的两个字符串相同,则它们就是乱序字符串。 \pause 

\lstinputlisting{code/anagramSolution2.py}  
\end{frame}

\begin{frame}\ft{\subsecname}

  你可能认为该算法的时间复杂度是 $O(n)$,因为只用了一个简单的循环来比较排序后的字符串。 \pause 

  但是,调用 Python 的排序函数是有开销的。我们在后面将会讲到,排序的时间复杂度通常是 $O(n^2)$ 或 $O(n\log n)$。\pause 


  所以排序操作比循环花费更多,从而该算法跟排序过程有同样的时间复杂度。
\end{frame}

\subsection{方案三: 穷举法}

\begin{frame}\ft{\subsecname}

  解决这类问题的一种强有力的方法是穷举所有可能性。 \pause 


  对于乱序检测,我们可以生成 $s_1$ 的所有乱序字符串列表,然后查看是不是有 $s_2$。这种方法有一点困难。 \pause 

  当 $s_1$ 生成所有可能的字符串时,第一个位置有 $n$ 种可能,第二个位置有 $n-1$ 种,第三个位置有 $n-2$ 种,等等。总数为 $n\cdot(n-1)\cdot(n-2)\cdots 3\cdot 2\cdot 1=n!$。虽然一些字符串可能是重复的,程序也不可能提前知道这样,所以他仍然会生成 $n!$ 个字符串。 \pause 

  事实上,$n!$ 比 $n^2$ 增长要快很多。如果 $s_1$ 有 $20$ 个字符长,则会产生 $20! = 2~432~902~008~176~640~000$ 个字符串。

  如果我们每秒处理一种可能字符串,那么需要 $77~146~816~596$ 年才能过完整个列表。所以这不是很好的解决方案。
\end{frame}

\subsection{方案四:计数和比较}

\begin{frame}\ft{\subsecname}
  最后一种解决方法是:利用两个乱序字符串具有相同数目的 $a, b, c$ 等字符的事实。
  \begin{enumerate}
  \item 首先计算的是每个字母出现的次数。由于有 $26$ 个可能的字符,我们就用 一个长度为 $26$ 的列表,每个可能的字符占一个位置。
  \item 每次看到一个特定的字符,就增加该位置的计数器。
  \item 最后如果两个列表的计数器一样,则字符串为乱序字符串。 
  \end{enumerate}
\end{frame}

\begin{frame}\ft{\subsecname}
  \lstinputlisting{code/anagramSolution4.py}
\end{frame}


\begin{frame}\ft{\subsecname}
  该方案仍有多次迭代,但与方案一不同,它不嵌套。

  前两个迭代的时间复杂度为$O(n)$, 第三个迭代用于比较两个计数列表,时间复杂度为$26$,共计$T(n)=2n+26$,即 $O(n)$。


  这样,我们就找到了一个线性量级的算法。

\end{frame}


\begin{frame}\ft{\subsecname}

  最后,我们来讨论下空间复杂度。虽然最后一个方案的时间复杂度是线性的,但它需要额外的存储来保存两个字符计数列表。
  换句话说,该算法\red{牺牲空间以换取时间}。 \pause 


  很多情况下,你需要在空间和时间之间做出权衡。
  上述程序中开辟的额外空间不大,但是如果有数百万个字符,就需要关注。\pause 

  作为一名计算机科学家,当给定一个特定的算法,将由你决定如何使用计算资源。
\end{frame}