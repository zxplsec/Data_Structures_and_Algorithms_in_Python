\subsection{向量范数}

%%******
\begin{frame}\ft{\subsecname}
\begin{dingyi}[向量范数]
向量范数是一个$\|\cdot\|:\R^n\to\R$的非负函数,它满足:\\
\begin{itemize}
\item[(1)] \textcolor{acolor4}{正定性}:
$$
\|x\|\ge 0, \quad \forall~x\in\R^n, ~~
\mbox{且} ~~
\|x\|=0 
~~\Longleftrightarrow~~ 
x=0
$$
\item[(2)] \textcolor{acolor4}{齐次性}:        
$$
\|\alpha x\|=|\alpha|\|x\|, \quad \forall~x\in\R^n,~\alpha\in\R
$$
\item[(3)] \textcolor{acolor4}{三角不等式}:
$$
\|x+y\|\le \|x\|+\|y\|, \quad \forall~x, y\in\R^n
$$
\end{itemize}
\end{dingyi}

\end{frame}

%%******
\begin{frame}\ft{\subsecname}
由(2)和(3)易知,
$\forall~x, y\in\R^n$有
$$
\big|\|x\|-\|y\|\big|\le \|x-y\|\le\max_{1\le i\le n}\|e_i\|\sum_{i=1}^n|x_i-y_i|.
$$
这说明\textcolor{acolor5}{$\|\cdot\|$作为$\R^n$的实函数是连续的}。
\end{frame}

%%******
\begin{frame}\ft{\subsecname}
\begin{dingyi}[$p$范数]
$$
\|x\|_p = \left(|x_1|^p+\cd+|x_n|^p\right)^{\frac1p}, ~ p\ge 1
$$
\begin{itemize}
\item $1$范数
$$
\|x\|_1 = |x_1|+\cd+|x_n|
$$
\item $2$范数
$$
\|x\|_2 = \left(|x_1|^2+\cd+|x_n|^2\right)^{\frac12} = \sqrt{x^T x}
$$
\item $\infty$范数
$$
\|x\|_{\infty} = \max_{i=1,\cd,n}\{|x_i|\}
$$
\end{itemize}
\end{dingyi}

\end{frame}

%%******
\begin{frame}\ft{\subsecname}
\begin{dingli}[范数等价性]
设$\|\cdot\|_{\alpha}$和$\|\cdot\|_{\beta}$是$\R^n$的任意两个范数,
则存在正常数$c_1$和$c_2$使得$\forall~x\in\R^n$有
$$
c_1\|x\|_{\alpha}\le\|x\|_{\beta}\le c_2\|x\|_{\alpha}
$$
\end{dingli}

$$
\begin{array}{ccccc}
\|x\|_{2} &\le& \|x\|_{1} &\le& \sqrt{n} \|x\|_{2},\\[0.2cm]
\|x\|_{\infty} &\le& \|x\|_{2} &\le& \sqrt{n} \|x\|_{\infty},\\[0.2cm]
\|x\|_{\infty} &\le& \|x\|_{1} &\le& n\|x\|_{\infty},\\
\end{array}
$$

\end{frame}

%%******
\begin{frame}\ft{\subsecname}
\begin{dingli}
设$x^{(k)},~x\in \R^n$,
则
$$\lim_{k\to\infty}\|x^{(k)}-x\|=0
~~\Longleftrightarrow~~
\lim_{k\to\infty} |x_i^{(k)}-x_i|=0,~i=1,\cd,n.
$$
\end{dingli}
\end{frame}

