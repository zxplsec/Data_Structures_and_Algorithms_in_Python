\section{0.2.~研究数值方法的必要性}

%%%%%%
\begin{frame}\ft{\secname}
\begin{small}
对于线性方程组
$$
Ax = b,
$$
\begin{dingli}[Crammer法则]
若$A$非奇异,则此方程组有唯一解,且
$$
x_i = \frac{|A_i|}{|A|}, ~~ i = 1, 2, \cd, n.
$$
其中$A_i$是将$A$的第$i$列换为$b$而得的矩阵。
\end{dingli}

\pause
该结论非常漂亮,它把线性方程组的求解问题归结为计算$n+1$个$n$阶行列式的计算问题。
\end{small}
\end{frame}

%%%%%%
\begin{frame}\ft{\secname}
\begin{small}
对于行列式的计算
\begin{dingli}[Laplace展开定理]
若$A$非奇异,则此方程组有唯一解,且
$$
|A| = a_{i1} A_{i1} + a_{i2} A_{i2} +  \cd + a_{in} A_{in}
$$
其中$A_{ij}$是元素$a_{ij}$的代数余子式。
\end{dingli}

\pause
该方法的运算量大的惊人,以至于完全不能用于实际计算。
\end{small}
\end{frame}

%%%%%%
\begin{frame}\ft{\secname}
\begin{small}
设$k$阶行列式所需乘法运算的次数为$m_k$,则
$$
m_k = k + k m_{k-1},
$$
于是有
$$
\begin{array}{ll}
m_n &= n + n m_{n-1} \\[0.2cm]
    &= n + n[(n-1) + (n-1) m_{n-2}] \\[0.2cm]
    &= \cd \\[0.2cm]
    &= n + n(n-1) + n(n-1)(n-2) + \cd + n(n-1)\cd 3 \cdot 2\\[0.2cm]
    &> n!
\end{array}
$$
故用Crammer法则和Laplace展开定理求解一个$n$阶线性方程组,所需乘法运算的次数就大于
$$
(n+1)n! = (n+1)!.
$$

\end{small}
\end{frame}

%%%%%%
\begin{frame}\ft{\secname}
\begin{small}
在一台百亿次的计算机上求解一个25阶线性方程组,则至少需要
$$
\frac{26!}{10^{10}\times 3600 \times 24 \times 365}
\approx
\frac{4.0329\times 10^{28}}{3.1526\times 10^{17}}
\approx
13\mbox{亿年}
$$
\pause
\vspace{0.1in}

而用下章介绍的消去法求解,则需要不到一秒钟。
\end{small}
\end{frame}


