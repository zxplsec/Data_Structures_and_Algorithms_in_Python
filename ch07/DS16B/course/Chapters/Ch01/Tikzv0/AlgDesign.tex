\documentclass{article}
\usepackage{CJK} 
\usepackage{graphics}
\usepackage{pgf}
\usepackage{tikz}
\usetikzlibrary{calc,shadows}
\usetikzlibrary{decorations.markings,scopes}
\usetikzlibrary{arrows,snakes,backgrounds,shapes}
\usetikzlibrary{decorations.pathmorphing}
\usepackage{listings}
\renewcommand{\ttdefault}{pcr}
\lstset{
  keywordstyle=\color{blue!70},
  frame=single,
  basicstyle=\ttfamily\bfseries\small,
  commentstyle=\small\color{red},
  rulesepcolor=\color{red!20!green!20!blue!20},
  tabsize=4,
  numbersep=5pt,
  %% backgroundcolor=\color{black!10},
  showspaces=false,
  showtabs=false,
  extendedchars=false,
  escapeinside=``,
  frame=no
}

\newcommand{\blue}{\textcolor{blue}}
\newcommand{\red}{\textcolor{red}}
\newcommand{\purple}{\textcolor{purple}}


\pgfrealjobname{survey}
\begin{document}
\begin{CJK}{UTF8}{gkai} 
  \beginpgfgraphicnamed{AlgDesign}
  \begin{tikzpicture}
    The graphic
    \node at (0,0)[fill=blue!20,draw,starburst,drop shadow,text width=0.8cm]{
      算法设计要求
    };
    %% \node at (0,2) [right,fill=red!20,draw,rectangle callout,callout relative pointer={(-0.6,-0.5)},rounded corners=3mm,text width=3cm]{ 
    \node at (1.5,0) [right,fill=red!20,draw,rectangle split, rectangle split parts=4,rounded corners=3mm,text width=7cm]{
      \blue{正确性}\\
      算法至少应该具有输入、输出和加工处理无歧义性,能正确反映问题的需求、能得到问题的正确答案。
      \begin{itemize}
      \item 无语法错误
      \item 对合法输入能产生满足要求的输出
      \item 对非法输入能给出满足规格的说明
      \item 对精心选择的甚至是刁难的测试数据都有满足要求的输出
      \end{itemize}
      \nodepart{second}
      \blue{可读性}\\
      便于阅读、理解和交流
      \nodepart{third}
      \blue{健壮性}\\
      当输入不合法时,也能做出相应处理,而不是产生异常或莫名其妙的结果
      \nodepart{fourth}
      \blue{时间效率高和存储量低}
    };   
  \end{tikzpicture}
  \endpgfgraphicnamed  
\end{CJK}

\end{document}


