% XeLaTeX can use any Mac OS X font. See the setromanfont command below.
% Input to XeLaTeX is full Unicode, so Unicode characters can be typed directly into the source.

% The next lines tell TeXShop to typeset with xelatex, and to open and save the source with Unicode encoding.

%!TEX TS-program = xelatex
%!TEX encoding = UTF-8 Unicode

\documentclass[10pt]{article}
\usepackage{geometry}                % See geometry.pdf to learn the layout options. There are lots.
\geometry{letterpaper}                   % ... or a4paper or a5paper or ... 
%\geometry{landscape}                % Activate for for rotated page geometry
%\usepackage[parfill]{parskip}    % Activate to begin paragraphs with an empty line rather than an indent
\usepackage{graphicx}
\usepackage{amssymb}
%%%% Package
\usepackage{amsmath,amssymb,amsthm}
\usepackage[UTF8,noindent]{ctex}
\usepackage{tikz}
\usetikzlibrary{calc}
\usetikzlibrary{arrows,snakes,backgrounds,shapes,shadows}
\usetikzlibrary{matrix,fit,positioning,decorations.pathmorphing}
\usepackage{listings}
\lstset{
  keywordstyle=\color{blue!70},
  frame=single,
  basicstyle=\ttfamily,
  commentstyle=\small\color{red},
  breakindent=0pt,
  rulesepcolor=\color{red!20!green!20!blue!20},
  rulecolor=\color{black},
  tabsize=4,
  numbersep=5pt,
  breaklines=true,
  %% backgroundcolor=\color{red!10},
  showspaces=false,
  showtabs=false,
  extendedchars=false,
  escapeinside=``,
  frame=no,
}

%%%% New Commands
\newcommand{\blue}{\textcolor{blue}}
\newcommand{\red}{\textcolor{red}}
\newcommand{\purple}{\textcolor{electricpurple}}
\newcommand{\ds}{\displaystyle}
\newcommand{\cd}{\cdots}
\newcommand{\dd}{\ddots}
\newcommand{\vd}{\vdots}
\newcommand{\id}{\iddots}


% Will Robertson's fontspec.sty can be used to simplify font choices.
% To experiment, open /Applications/Font Book to examine the fonts provided on Mac OS X,
% and change "Hoefler Text" to any of these choices.

\usepackage{fontspec,xltxtra,xunicode}
\defaultfontfeatures{Mapping=tex-text}
\setromanfont[Mapping=tex-text]{Hoefler Text}
\setsansfont[Scale=MatchLowercase,Mapping=tex-text]{Gill Sans}
\setmonofont[Scale=MatchLowercase]{Andale Mono}

\begin{document}
\renewcommand{\proofname}{\textbf{证明}}
%%%% New Theorem
\newtheorem{li}{例}
\newtheorem{jielun}{结论}
\newtheorem{dingli}{定理}
\newtheorem{mingti}{{命题}} 
\newtheorem{yinli}{{引理}} 
\newtheorem{tuilun}{{推论}}
\newtheorem{dingyi}{{定义}} 
\newtheorem*{jie}{{解}}
\newtheorem*{zhengming}{{证明}}
\newtheorem{zhu}{{注}}
\newtheorem*{zhu*}{{注}}
\newtheorem{xingzhi}{{性质}}
\newtheorem{wenti}{{问题}}

\title{哈夫曼树(Huffman Tree)}
\author{张晓平}
%\date{}                                           % Activate to display a given date or no date
\maketitle

\section{定义}
哈夫曼树是一种带权路径长度最短的二叉树,也称最优二叉树。


\section{构造}
构造过程如下:
\begin{enumerate}
\item 将所有左、右子树都为空的结点作为根结点;
\item 选出两颗根结点的权值最小的树作为一棵新树的左、右子树,且置新树的根结点的权值为其左、右子树上根结点的权值之和。\red{注:左子树的权值应小于右子树的权值。}
\item 从森林中删除这两棵树,同时把新树加入到森林中。
\item 重复2、3,直到森林中只有一棵树为止,此树便是Huffman树。
\end{enumerate}


\section{Huffman编码}
利用Huffman树求得的用于通信的二进制编码称为Huffman编码。树中从根到每个叶子结点都有一条路径,对路径上各分支作如下约定:指向左子树的分支表示0码,指向右子树的分支表示1码。取每条路径上的0或1的序列作为各叶子结点对应的字符编码,即是Huffman编码。


\section{程序实现}
\lstinputlisting[language=C,frame=single]{Huffman/Huffman.h}

\lstinputlisting[language=C,frame=single]{Huffman/Huffman.c}

\lstinputlisting[language=C,frame=single]{Huffman/HuffmanTest.c}


\begin{lstlisting}[title=运行结果,frame=single]
input character:
aaaabbbcccddddddeeeeeffffffggggggg

Depth of Huffman Tree is 5

Huffman Codes are:
d:6 00
g:7 01
a:4 100
e:5 101
f:6 111
c:3 1100
b:3 1101
\end{lstlisting}

\end{document}  