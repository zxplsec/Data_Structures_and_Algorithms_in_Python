\section{线性表}

\begin{frame}

  
  线性表是一种典型的线性结构,其中的数据元素是\red{有序且有限}, 并且
  \begin{itemize}
  \item
    有唯一首元;
  \item
    有唯一末元;
  \item
    除首元外,每个元素均有唯一的直接前驱;
  \item
    除末元外,每个元素均有唯一的直接后继. 
  \end{itemize}•

\end{frame}


\subsection{线性表的逻辑结构}
\begin{frame}\ft{\secname}
\begin{dingyi}[线性表 Link List]
由$n$个数据元素$a_1,a_2,\cd,a_n$组成的有限序列.
\begin{itemize}
\item $a_i$的数据类型相同;
\item $n$称为线性表的长度.
\end{itemize}
\end{dingyi}

\pause 
\begin{enumerate}
\item $n\ge 0$, 数据元素又称结点;
\item $n=0$为空表;\\[0.1in]
\item $n>0$为非空的线性表, 记为$(a_1,a_2,\cd,a_n)$. \\[0.1in]
\item[$\diamond$]
$a_1$称为首结点, $a_n$称为尾结点.  
\item[$\diamond$]
$a_1,a_2,\cd,a_{i-1}$都是$a_i%(2\le i \le n)
$的前驱, 其中$a_{i-1}$是$a_i$的直接前驱. 
\item[$\diamond$]
$a_{i+1},a_{i+2},\cd,a_{n}$都是$a_i%(1\le i \le n-1)
$的后继, 其中$a_{i+1}$是$a_i$的直接后继. 
\end{enumerate}•
\end{frame}

\begin{frame}\ft{线性表的逻辑结构}
%线性表中的数据元素$a_i$所代表的具体含义随具体应用的不同而不同, 
%只不过是一个抽象的表示符号. 
\begin{itemize}
\item%[$\spadesuit$]
结点可以是单值元素%(每个元素只有一个数据项)
\end{itemize}

\begin{li}
字母表:(A,~B,~C,~$\cd$,~Z)
\end{li}	

\begin{li}
扑克点数:(2,~3,~4,~$\cd$,~J,~Q,~K,~A)
\end{li}	

\end{frame}

\begin{frame}\ft{线性表的逻辑结构}
\begin{itemize}
\item%[$\spadesuit$]
结点可以是记录型元素. 
\item[] 每个元素可含多个数据项, 每一项称为结点的一个域. 
每个元素有一个可以唯一标识每个结点的域, 称为\red{关键字}. 
\end{itemize}

\begin{exampleblock}{例3}
某校2014级同学的基本情况:
$$
\begin{array}{ccc}
\{&\mbox{('20140212001','张三','男',06/24/1992),}&\\[0.1cm]
&\mbox{('20140212002','李四','男',07/20/1992),}&\\[0.1cm]
&\cd&\\[0.1cm]
&\mbox{('20140212090','王五','男',04/05/1992)}&\}
\end{array} 
$$
\end{exampleblock}	

\end{frame}

\begin{frame}\ft{线性表的逻辑结构}
\begin{itemize}
\item%[$\spadesuit$]
若结点按值从小到大(或从大到小)排列, 
称线性表是\red{有序}的. \\[0.2in]
\item%[$\spadesuit$]
线性表的长度可根据需要增长或缩短. \\[0.2in]
\item%[$\spadesuit$]
可对结点进行访问、插入和删除操作. 
\end{itemize}

\end{frame}


\subsection{线性表的抽象数据类型定义}
\begin{frame}[fragile]
\begin{lstlisting}[mathescape=true]
ADT List{
  `数据对象:` $\mathrm D=\{a_i|a_i\in \mathrm{ElemSet},i=1,2,\cd,n,n\ge0\}$
  `数据关系:` $\mathrm R=\{\langle a_{i-1},a_i\rangle|a_{i-1},a_i\in \mathrm D, i=2,3,\cd,n\}$
  `基本操作:`
  InitList(&L)
  `操作结果:构造一个空的线性表L`
  ListLength(L)
  `初始条件:线性表L已存在`
  `操作结果:若L为空表, 则返回0, 否则返回线性表的长度`
  ...
  GetElem(L,i,&e)
  `初始条件:线性表L已存在,`$1\le i \le \mathrm{ListLength(L)}$
  `操作结果:用e返回L中第i个数据元素的值`
  ListInsert(L,i,&e)
  `初始条件:线性表L已存在,`$1\le i \le \mathrm{ListLength(L)}$
  `操作结果:在线性表L中的第i个位置插入元素e`
  ...
} ADT List
\end{lstlisting}
\end{frame}

