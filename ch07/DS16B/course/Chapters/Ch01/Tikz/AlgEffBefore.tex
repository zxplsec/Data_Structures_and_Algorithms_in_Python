\begin{figure}
  
  \centering
    \begin{tikzpicture}[node distance=6cm]
  \node [fill=blue!20,draw,starburst,drop shadow,text width=5cm] (1) {
  \blue{\large 事前分析估算方法}\\
  在编制程序前,依据统计方法对算法进行估算。
  };
  \pause
  
  \node[below left of=1,rectangle split,rectangle split parts=5,rounded corners=3mm,draw,text width=7cm,fill=red!20] (2){
    \blue{程序运行时间取决于}
      \nodepart{two}
      (1)~算法采用的策略、方法 (\red{算法好坏的根本})
      \nodepart{three}
      (2)~编译产生的代码质量 (\red{软件支持})
      \nodepart{four}
      (3)~问题的输入规模 
      \nodepart{five}
      (4)~机器执行指令的速度 (\red{硬件性能})
    };
  \pause
    \node[right of=2,rectangle split,ellipse,draw,text width=2.5cm,fill=red!20] {
      程序的运行时间,依赖于算法的好坏和问题的输入规模。
    };

  \end{tikzpicture}

\end{figure}
