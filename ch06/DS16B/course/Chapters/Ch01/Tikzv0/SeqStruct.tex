\documentclass{article}
\usepackage{CJK} 
\usepackage{graphics}
\usepackage{pgf}
\usepackage{tikz}
\usetikzlibrary{calc,shadows}
\usetikzlibrary{decorations.markings,scopes}
\usetikzlibrary{arrows,snakes,backgrounds,shapes}
\usetikzlibrary{decorations.pathmorphing}
\newcommand{\blue}{\textcolor{blue}}
\newcommand{\red}{\textcolor{red}}
\newcommand{\purple}{\textcolor{purple}}


\pgfrealjobname{survey}
\begin{document}
\begin{CJK}{UTF8}{gkai} 
  \beginpgfgraphicnamed{SeqStruct}
  \begin{tikzpicture}[scale=1.5]
    \tikzstyle{state}=[draw,circle]

        % The graphic
    \node at (0,2)[right,fill=blue!20,draw,starburst,drop shadow,text width=6.5cm]{
      \blue{\large 顺序存储结构} \\
      把数据元素存放在连续的存储单元里,其数据间的逻辑关系与物理关系一致。
    } ;

    %% \node at (-1,-2) [fill=blue!20,draw,ellipse callout, callout relative pointer={(0.,0.7)},text width=5.5cm]{ 
    %%   数据的存储结构应正确反映数据元素之间的逻辑关系。如何存储数据元素之间的逻辑关系,是实现物理结构的重点和难点。
    %% };

    %% \node at (2.5,-1.3) [text width=1.5cm,decorate,decoration=saw,fill=blue!20,draw,circle,text=red]{
    %%   顺序存储 链式存储
    %% };
    \def\xx{0.8}
    \draw[very thick] (0,0)--(9*\xx,0);
    \foreach \i in {0,...,9} 
    \draw[very thick] (\i*\xx,0)--(\i*\xx,1*\xx);      
    \foreach \i in {1,...,9} 
    \draw[thick] (\i*\xx-0.5*\xx,0.5*\xx)node[]{\i}circle(0.5*\xx);

    
  \end{tikzpicture}
  \endpgfgraphicnamed  
\end{CJK}

\end{document}


