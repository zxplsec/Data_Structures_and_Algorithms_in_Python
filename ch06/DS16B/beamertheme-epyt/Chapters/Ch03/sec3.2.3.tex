\subsection{链队列}

\begin{frame}\ft{\subsecname}
\begin{dingyi}
队列的链式存储结构,就是线性表的单链表,只不过它只能尾进头出,简称为链队列。
\end{dingyi}
\end{frame}

\begin{frame}\ft{\subsecname}
\begin{zhu}
  为操作方便,通常将队头指针指向链队列的头结点,而队尾指针指向终端结点。
\end{zhu}
\pause

\begin{figure}
\centering
\begin{tikzpicture}
\def\x{1}
\def\xx{0.75*\x}
\def\y{0.75}

\foreach \c in {0} {
\foreach \r in {0,2,4,6} {
\ifthenelse{\r=4}{
  \node [] at(\r*\x+0.5*\x,\c*\y+0.5*\y){$\cd$};
}{        
  \ifthenelse{\r=0}{
  \draw[thick,fill=green!10,rounded corners] (\r*\x,\c*\y)rectangle(\r*\x+1.5*\x,\c*\y+\y);
  \draw[thick] (\r*\x+\x,\c*\y)--(\r*\x+\x,\c*\y+\y);
  }{
  \draw[thick,fill=red!10,rounded corners] (\r*\x,\c*\y)rectangle(\r*\x+1.5*\x,\c*\y+\y);
  \draw[thick] (\r*\x+\x,\c*\y)--(\r*\x+\x,\c*\y+\y);
  }
}
%% 
\draw[->,very thick] (\r*\x+1.25*\x,\c*\y+0.5*\y)--(\r*\x+1.9*\x,\c*\y+0.5*\y);

%% 
\ifthenelse{\r=0}{
  \draw[->,very thick](\r*\x+0.75*\xx,\c*\y-1.0*\y)node[below]{{\tt front}}--(\r*\x+0.75*\xx,\c*\y-0.2*\y);
  \draw[snake=brace,raise snake=3pt,thick] (\r*\x+0.0*\x,\c*\y+1.0*\y)--node[above=5pt]{\footnotesize{头结点}}(\r*\x+1.5*\x,\c*\y+1.0*\y);
}{}
\ifthenelse{\r=2}{
  \node [] at(\r*\x+0.5*\xx,\c*\y+0.5*\y){$a_1$};
  \draw[snake=brace,raise snake=3pt,thick] (\r*\x+0.0*\x,\c*\y+1.0*\y)--node[above=5pt]{\footnotesize{队头}}(\r*\x+1.5*\x,\c*\y+1.0*\y);
}{}
\ifthenelse{\r=6}{
  \node [] at(\r*\x+0.5*\xx,\c*\y+0.5*\y){$a_n$};
  \filldraw[black!80] (\r*\x+2*\x,\c*\y+0.25*\y)rectangle(\r*\x+2.5\x,\c*\y+0.75*\y);
  \draw[->,very thick](\r*\x+0.75*\xx,\c*\y-1.0*\y)node[below]{{\tt rear}}--(\r*\x+0.75*\xx,\c*\y-0.2*\y);
  \draw[snake=brace,raise snake=3pt,thick] (\r*\x+0.0*\x,\c*\y+1.0*\y)--node[above=5pt]{\footnotesize{队尾}}(\r*\x+1.5*\x,\c*\y+1.0*\y);
}{}

}
}
\end{tikzpicture}

%% \caption{}  
\end{figure}

\end{frame}

\begin{frame}\ft{\subsecname}
\begin{zhu}
队列为空时,{\tt front}和{\tt rear}都指向头结点。  
\end{zhu}

\pause
\begin{figure}
\centering
\begin{tikzpicture}
\def\x{1}
\def\xx{0.75*\x}
\def\y{0.75}

\foreach \c in {0} {
\foreach \r in {0} {
\draw[thick,fill=green!10,rounded corners] (\r*\x,\c*\y)rectangle(\r*\x+1.5*\x,\c*\y+\y);
\draw[thick] (\r*\x+\x,\c*\y)--(\r*\x+\x,\c*\y+\y);

\draw[->,very thick](\r*\x-1.2*\x,\c*\y+0.5*\y)--node[above]{{\tt front}}(\r*\x-0.2*\x,\c*\y+0.5*\y);
\draw[snake=brace,raise snake=3pt,thick] (\r*\x+0.0*\x,\c*\y+1.0*\y)--node[above=5pt]{\footnotesize{头结点}}(\r*\x+1.0*\x,\c*\y+1.0*\y);
\draw[->,very thick](\r*\x+0.5*\xx,\c*\y-1.0*\y)node[below]{{\tt rear}}--(\r*\x+0.5*\xx,\c*\y-0.2*\y);
}
}
\end{tikzpicture}

%% \caption{}  
\end{figure}

\end{frame}




\begin{frame}[fragile,allowframebreaks]\ft{\tt LinkQueue.h}
\lstinputlisting[
title={},
language=C,
]{Chapters/Ch03/Code/LinkQueue/LinkQueue.h}
\end{frame}


\begin{frame}\ft{\subsecname}
入队操作,就是在链表尾部插入结点。\pause 
\input{Chapters/Ch03/Tikz/LinkQueue/EnLinkQueue}
\end{frame}



\begin{frame}\ft{\tt Enter.c}
\lstinputlisting[
language=C,
]{Chapters/Ch03/Code/LinkQueue/Enter.c}

\end{frame}

\begin{frame}\ft{\subsecname}
出队操作,就是头结点的后继结点出队,将头结点的后继改为它后面的结点。\pause 
\input{Chapters/Ch03/Tikz/LinkQueue/DeLinkQueue}
\end{frame}

\begin{frame}\ft{\subsecname}
若链表处头结点外只剩一个元素时,则需将{\tt rear}指向头结点。
\begin{figure}
\centering
\begin{tikzpicture}
\def\x{1.2}
\def\xx{0.75*\x}
\def\y{0.75}

\foreach \c in {0} {
\foreach \r in {0,2} {
\ifthenelse{\r=6}{
  \node [] at(\r*\x+0.5*\x,\c*\y+0.5*\y){$\cd$};
}{          
  \draw[thick] (\r*\x,\c*\y)rectangle(\r*\x+\xx,\c*\y+\y);
  \draw[thick] (\r*\x+\xx,\c*\y)rectangle(\r*\x+\x,\c*\y+\y);
}
%% 
\ifthenelse{\r=2}{
  \filldraw[thick] (\r*\x+\xx,\c*\y)rectangle(\r*\x+\x,\c*\y+\y);
}{
  \draw[->,very thick] (\r*\x+0.875*\x,\c*\y+0.5*\y)--(\r*\x+1.9*\x,\c*\y+0.5*\y);
}
%% 
\ifthenelse{\r=0}{
  \draw[->,very thick](\r*\x-1.2*\x,\c*\y+0.5*\y)--node[above]{{\tt front}}(\r*\x-0.2*\x,\c*\y+0.5*\y);
}{}
\ifthenelse{\r=2}{
  \node [] at(\r*\x+0.5*\xx,\c*\y+0.5*\y){$a_1$};
}{}

\ifthenelse{\r=4}{
  \node [] at(\r*\x+0.5*\xx,\c*\y+0.5*\y){$a_2$};
}{}        
}


\pause 
\def\r{2}
\draw[snake=brace,raise snake=3pt,very thick](\r*\x+0.0*\x,\c*\y+1.0*\y)--node[above=5pt]{{\tt p=front->next}}node[above=15pt]{{\tt (1)}}
(\r*\x+1.0*\x,\c*\y+1.0*\y);

\pause 
\def\r{1}
\node [] at(\r*\x+0.5*\xx,\c*\y+0.5*\y){\includegraphics[width=0.5cm]{Chapters/Ch03/Tikz/cross.jpg}};


\pause 
\def\r{0}
\filldraw[thick] (\r*\x+\xx,\c*\y)rectangle(\r*\x+\x,\c*\y+\y);
\draw[->,thick]  (\r*\x+0.5*\xx,\c*\y-1.1*\y) node[below]{{\tt rear}}--node[left]{{\tt (3)}} (\r*\x+0.5*\xx,\c*\y-0.1*\y) ;

}
\end{tikzpicture}

%% \caption{}  
\end{figure}

\end{frame}

\begin{frame}\ft{\tt Exit.c}
\lstinputlisting[
language=C,
]{Chapters/Ch03/Code/LinkQueue/Exit.c}  
\end{frame}

\begin{frame}\ft{\subsecname}
循环队列和链队列的比较:

\begin{itemize}
\item 从时间上看,其基本操作的复杂度均为$O(1)$,不过循环队列事先申请好空间,使用期间不释放,而对于链队列,每次申请和释放结点会存在一些时间开销。若入队出队频繁,两者会有细微差异。\\[0.1in]
\item 从空间上看,循环队列须有一个固定长度,故存在存储元素个数和空间浪费的问题。而链队列不存在该问题,尽管它需要一个指针域,会产生一些空间上的开销,但也可以接受。所以在空间上,链表更加灵活。
\end{itemize}
\end{frame}

\begin{frame}\ft{\subsecname}
\begin{itemize}
\item 在可以确定队列长度最大值的情况下,建议使用循环队列;\\[0.1in]
\item 若无法预估队列长度,请使用链队列。
\end{itemize}
\end{frame}






