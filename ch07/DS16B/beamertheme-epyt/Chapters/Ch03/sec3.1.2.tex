\subsection{顺序栈}

% \begin{frame}\ft{\subsecname}
%   既然栈是线性表的特例,那么栈的顺序存储其实是线性表顺序存储的简化。
%   栈的顺序存储结构简称顺序栈,。根据数组是否可以根据需要增大,又可分为静态顺序栈和动态顺序栈。
% \begin{itemize}
% \item[$\diamond$]
% 静态顺序栈实现简单,但不能根据需要增大栈的存储空间;
% \item[$\diamond$]
% 动态顺序栈可以根据需要增大栈的存储空间,但实现稍微复杂。
% \end{itemize}

% \end{frame}

\begin{frame}\ft{\subsecname}
  既然栈是线性表的特例,那么栈的顺序存储其实是线性表顺序存储的简化。
  栈的顺序存储结构简称顺序栈,用数组来实现。 \vskip.1in

  \pause
  \begin{wenti}
    对于栈,用数组哪一端作为栈顶或栈底会比较好?
  \end{wenti} \vskip.1in

  \pause
  下标为0的一端作为栈底比较好,因为首元素都存在栈底,变化最小。
\end{frame}
%
%
\begin{frame}[fragile]\ft{\subsecname}
  \begin{lstlisting}[title=栈的结构定义,language=C]
    #define MAXSIZE 100
    typedef int ElemType;
    typedef struct SqStack{
      ElemType data[MAXSIZE];
      int top;
    }SqStack;
  \end{lstlisting}

  \begin{itemize}
  \item {\tt top}用于指示栈顶元素在数组中的位置,它必须小于存储栈的长度{\tt StackSize};
  \item 当栈存在一个元素时,{\tt top == 0},因此通常把空栈的判定条件定为{\tt top == -1}.
  \end{itemize}
\end{frame}
%
%
\begin{frame}\ft{\subsecname}
  \input{Chapters/Ch03/Tikz/SqStack/SqStack}
\end{frame}
%
\begin{frame}\ft{\subsecname}
  \input{Chapters/Ch03/Tikz/SqStack/PushSqStack}
\end{frame}
%
\begin{frame}[fragile]\ft{\subsecname}
  \lstinputlisting[
    title={\tt Push.c},
    language=C,
  ]{Chapters/Ch03/Code/SqStack/Push.c}
\end{frame}

\begin{frame}\ft{\subsecname}
  \input{Chapters/Ch03/Tikz/SqStack/PopSqStack}
\end{frame}

\begin{frame}[fragile]\ft{\subsecname}
  \lstinputlisting[
    title={\tt Pop.c},
    language=C,
  ]{Chapters/Ch03/Code/SqStack/Pop.c}
\end{frame}


\begin{frame}[fragile]
\begin{center}
  \textcolor{acolor5}{\Large 顺序栈之完整程序}
\end{center}
\end{frame}
\begin{frame}[fragile,allowframebreaks]\ft{\tt SqStack.h}
  \lstinputlisting[
    language=C,
  ]{Chapters/Ch03/Code/SqStack/SqStack.h}
\end{frame}

\begin{frame}[fragile]\ft{\tt Init.c}
  \lstinputlisting[
    language=C,
  ]{Chapters/Ch03/Code/SqStack/Init.c}
\end{frame}

\begin{frame}[fragile]\ft{\tt PrintS.c}
  \lstinputlisting[
    language=C,
  ]{Chapters/Ch03/Code/SqStack/PrintS.c}
\end{frame}

\begin{frame}[fragile]\ft{\tt Push.c}
  \lstinputlisting[
    language=C,
  ]{Chapters/Ch03/Code/SqStack/Push.c}
\end{frame}

\begin{frame}[fragile]\ft{\tt Pop.c}
  \lstinputlisting[
    language=C,
  ]{Chapters/Ch03/Code/SqStack/Pop.c}
\end{frame}

\begin{frame}[fragile]\ft{\tt Clear.c}
  \lstinputlisting[
    language=C,
  ]{Chapters/Ch03/Code/SqStack/Clear.c}
\end{frame}

\begin{frame}[fragile]\ft{\tt Destroy.c}
  \lstinputlisting[
    language=C,
  ]{Chapters/Ch03/Code/SqStack/Destroy.c}
\end{frame}

