\section{为什么要学习数据结构和抽象数据类型}
\begin{frame}\ft{\secname}
  为管理问题的复杂性及解决过程,计算机科学家使用抽象使他们能够专注于 “大局” 而不会迷失在细节中。\vskip.1in

  通过对问题进行建模,我们能够更好和更有效地解决问题。\vskip.1in


  这些模型允许我们以更加一致的方式来描述我们的算法。
\end{frame}

\begin{frame}\ft{\secname}
  \begin{itemize}
  \item \blue{过程抽象}:隐藏特定函数的细节,以允许用户或客户端在高层查看它。\\[0.1in] \pause 
  \item \blue{数据抽象,即抽象数据类型(ADT)}:对数据和允许操作的逻辑描述,不用考虑如何实现它们。
  \item[] 这意味着我们只关心数据表示什么,而不关心它最终将如何构造。通过提供这种级别的抽象,我们围绕数据创建一个封装。通过封装实现细节,我们将它们从用户的视图中隐藏。这称为信息隐藏。
  \end{itemize}

\end{frame}

\begin{frame}\ft{\secname}

\begin{figure}[htbp]
  \centering
  \includegraphics[width=4in]{images/ds_adt.png}
  \caption{展示了ADT是什么以及如何操作。用户与接口交互,使用抽象数据类型指定的操作。抽象数据类型是用户与之交互的 shell。实现隐藏在更深的底层。用户不关心实现的细节。}
\end{figure}
\end{frame}

\begin{frame}\ft{\secname}
  ADT的实现要求我们使用一些程序构建和原始数据类型的集合来提供数据的物理视图。

  “逻辑”与“物理”两个视角的分离,允许我们将问题定义复杂的数据模型,而不给出关于模型如何实际构建的细节。
  这提供了独立于实现的数据视图。


  由于通常有许多不同的方法来实现抽象数据类型,所以这种实现独立性允许程序员在不改变数据的用户与其交互的方式的情况下切换实现的细节。

  用户可以继续专注于解决问题的过程。
\end{frame}

